\documentclass[a4paper,8pt]{article}
%-----------------------------------------------------------
\usepackage[top=0.1in, left=0.3in, bottom = 0.25cm, right=0.3in]{geometry}
\usepackage{graphicx}
\usepackage{url}
\usepackage{palatino}	
\usepackage{tabularx}
\usepackage{enumitem}
\fontfamily{SansSerif}
\selectfont

\usepackage{hyperref}
\hypersetup{
    colorlinks=true,
    linkcolor=blue,
    filecolor=magenta,      
    urlcolor=black,
}

\urlstyle{same}


\usepackage[T1]{fontenc}
\usepackage
%[ansinew]
[utf8]
{inputenc}

\usepackage{color}
\usepackage{tcolorbox}
\definecolor{mygrey}{gray}{0.75}
% \textheight=9.75in
\raggedbottom

\setlength{\tabcolsep}{0in}
\newcommand{\isep}{-2 pt}
\newcommand{\lsep}{-0.5cm}
\newcommand{\psep}{-0.6cm}
\renewcommand{\labelitemii}{$\circ$}

\pagestyle{empty}
%-----------------------------------------------------------
%Custom commands
\newcommand{\resitem}[1]{\item #1 \vspace{-2pt}}
\newcommand{\resheading}[1]{\begin{tcolorbox} \begin{center} #1 \end{center} \end{tcolorbox}}
\newcommand{\ressubheading}[3]{
\begin{tabular*}{6.62in}{l @{\extracolsep{\fill}} r}
	\textsc{{\textbf{#1}}} & \textsc{\textit{[#2]}} \\
\end{tabular*}\vspace{-8pt}}
%-----------------------------------------------------------

\begin{document}
% \hspace{0.5cm}\\[-0.2cm]

\textbf{\large{Rohan Chandra}} 
% \indent 9210 Rhode Island Ave, \\
% \indent College Park, MD, 20740  \\
\indent Email : \textbf{rohan@cs.umd.edu}  \ homepage: \textbf{https://rohanchandra30.github.io/}\\
% \indent Mobile No.: \textbf{+1 2404475891} \\

\resheading{\textbf{ACADEMIC DETAILS} }
%\begin{table}[ht!]
%\begin{center}
\indent \begin{tabular}{ c @{\hskip 1in} c @{\hskip 1in} c @{\hskip .5in} c @{\hskip .5in} c @{\hskip 0.4in} }
% \bigskip
\textbf{Program} & \multicolumn{1}{c}{\textbf{University}} &  & \textbf{GPA(\small{Last Two Years})} & \textbf{Class High}\\
M.S. in CS, 2018\,\, & \multicolumn{1}{c}{University of Maryland} & & {3.837	} & {4.00}\\
B.Tech in ECE, 2016 & \multicolumn{1}{c}{Delhi Technological University} & & {80.00} & {84.00}\\
% \hline
\end{tabular}
% \bigskip
% \newline
% \textit{Undergraduate Thesis:} A Novel Architecture For A Band-Stop Notch Filter. \href{http://search.proquest.com/openview/324892b65f61b2bd4cee1526c67b0ef8/1?pq-origsite=gscholar&cbl=2030624}{\textbf{Paper}}
%\end{center}
%\end{table}


\resheading{\textbf{RELEVANT CLASSES} }
%\begin{table}[ht!]
%\begin{center}
\indent \begin{tabular}{ l @{\hskip 1.1in} l @{\hskip 1.1in} l @{\hskip 1.1in}  }
% \bigskip
% \hline
\textbf{PhD level} & \textbf{Text(coverage)} & \textbf{Masters level}   \\
% \hline
Optimization & \textit{Boyd/Research Papers} & Linear Algebra  \\
Machine Learning & \textit{UML, FML, Murphy} & Prob. and Stats  \\
Computer Vision & \textit{Research Papers} &  Statistics I\\
Spectral Methods and Reinforcement Learning & \textit{Research Papers}
% \hline
\end{tabular}


\resheading{\textbf{PUBLICATIONS} }
\begin{itemize}[noitemsep]
\item \noindent \textbf{Rohan Chandra}, Ziyuan Zhong, Justin Hontz, Val McCulloch, Christoph Studer, Tom Goldstein, \textit{Phasepack User Guide}, arXiv Preprint, 2017
\item \noindent \textbf{Rohan Chandra}, Ziyuan Zhong, Justin Hontz, Val McCulloch, Christoph Studer, Tom Goldstein, \textit{Phasepack: A Phase Retrieval Library}, Submitted to IEEE Proceedings of Asilomar Conference on Signals, Signals, Systems and Computers, 2017.
\item \noindent Arthur Benjamin, \textbf{Rohan Chandra}, \textit{Multiply by 9}, The College Mathematics Journal, 2016.
\item \noindent Rashika Anurag, Neeta Pandey, \textbf{Rohan Chandra}, Rajeshwari Pandey, \textit{Voltage Mode Second Order Notch/All-Pass Filter Realization Using OTRA}, i-Manager's Journal on Electronics Engineering, 2016. \textit{Undergraduate Thesis}
\end{itemize}

% \resheading{\textbf{RESEARCH EXPERIENCE} }
% \begin{itemize}[noitemsep]
% \item \textbf{Phase Retrieval:} Created PhasePack, a library for various classical and contemporary phase retrieval algorithms. PhasePack's purpose is to create a common interface for a wide range of phase retrieval schemes, and to provide a common test bed using both syntheitc and empirical imaging datasets.

% \end{itemize}


\resheading{\textbf{PROJECTS} }
\begin{itemize}[noitemsep]
\item \textbf{Texture Synthesis using Stacked VAE's:} Based on the success of DRAW - a generative model to create images, I am extending the concept to create textures.

\item\textbf{Autonomous Vehicles:} Implemented the lane detection module and helped engineer a joystick enabled 3-wheeler. Also worked in navigation and localisation. 

\item \textbf{Structure from Motion:} Wrote code from scratch and successfully reconstructed a 3-D scene from multiple images using non-linear optimization of feature point triangulation, PnP, and finally bundle adjustment. Received highest points for this project.
\end{itemize}



% \end{itemize}

\resheading{\textbf{TEACHING EXPERIENCE} }
\begin{itemize}[noitemsep]
\item \noindent Discrete Mathematics (Fall 2017) (Recitation, Office Hours, Grading)
\item \noindent Intro to Object Oriented Programming (Spring 2017) (Office Hours, Grading)
\item \noindent Computer Networks (Fall 2016) (Office Hours, Grading)
\end{itemize}


\resheading{\textbf{STRENGTHS \& SKILLS} }
\begin{itemize}[noitemsep]
\item \textbf{Top Writer} in the category of mathematical optimization on Quora.

\item \textbf{Mental Math, Speed Math}\\
 \emph{Links for these articles can be found on my homepage} \\[-0.6cm]\\
	\begin{itemize}\itemsep \isep
	\item Faster method to mentally multiply numbers by 9, 99, 999... and so on (method co-authored with \href{https://www.ted.com/talks/arthur_benjamin_does_mathemagic}{\textbf{Dr. Arthur Benjamin}}.
	\item Alternate method to square two digit numbers.
	\item Generalizing the "Find The Missing Digit Trick!"
	\end{itemize}
\end{itemize}

\begin{itemize}
\item \noindent \textbf{Chess}: State Level Champion
\end{itemize}

\resheading{\textbf{DEPARTMENTAL SERVICE} }
\begin{itemize}
\item \noindent Part of the fall 2017 \textbf{review committee} at UMD that screens applications for the Masters in CS program.
\end{itemize}




 % \emph{You can read about these in my \href{https://rohanchandra306.wordpress.com/}{\textbf{blog}}} \\[-0.6cm]
\end{document}

